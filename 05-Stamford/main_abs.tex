\documentclass[a4paper,UKenglish]{oasics-v2016}
%This is a template for producing OASIcs articles. 
%See oasics-v2016-manual.pdf for further information.
%for A4 paper format use option "a4paper", for US-letter use option "letterpaper"
%for british hyphenation rules use option "UKenglish", for american hyphenation rules use option "USenglish"
% for section-numbered lemmas etc., use "numberwithinsect"
 
\usepackage{microtype}%if unwanted, comment out or use option "draft"
\usepackage{graphicx}
\graphicspath{ {Plots/} }

%\graphicspath{{./graphics/}}%helpful if your graphic files are in another directory

\bibliographystyle{plainurl}% the recommended bibstyle

% Author macros::begin %%%%%%%%%%%%%%%%%%%%%%%%%%%%%%%%%%%%%%%%%%%%%%%%
\title{Discriminative and Generative models for clinical risk estimation: An empirical comparison}
\titlerunning{Discriminative and Generative models for clinical risk estimation} %optional, in case that the title is too long; the running title should fit into the top page column

%% Please provide for each author the \author and \affil macro, even when authors have the same affiliation, i.e. for each author there needs to be the  \author and \affil macros
\author[1]{John Stamford}
\author[2]{Chandra Kambhampati}
\affil[1]{Department of Computer Science, The University of Hull, Kingston upon Hull, HU6 7RX\\
\texttt{j.stamford@2014.hull.ac.uk}}
\affil[2]{Department of Computer Science, The University of Hull, Kingston upon Hull, HU6 7RX\\
\texttt{c.kambhampati@hull.ac.uk}}


\authorrunning{John Stamford, Chandra Kambhampati} %mandatory. First: Use abbreviated first/middle names. Second (only in severe cases): Use first author plus 'et. al.'

\Copyright{John Stamford}%mandatory, please use full first names. OASIcs license is "CC-BY";  http://creativecommons.org/licenses/by/3.0/

\subjclass{ D.4.8 Performance }% mandatory: Please choose ACM 1998 classifications from http://www.acm.org/about/class/ccs98-html . E.g., cite as "F.1.1 Models of Computation". 
\keywords{Discriminative, Generative, Naïve Bayes, Logistic Regression, Clinical Risk}% mandatory: Please provide 1-5 keywords
% Author macros::end %%%%%%%%%%%%%%%%%%%%%%%%%%%%%%%%%%%%%%%%%%%%%%%%%

%Editor-only macros:: begin (do not touch as author)%%%%%%%%%%%%%%%%%%%%%%%%%%%%%%%%%%
\EventEditors{Fergus Leahy}
\EventNoEds{2}
\EventLongTitle{Discriminative and Generative models for clinical risk estimation: An empirical comparison}
\EventShortTitle{ICCSW 2017}
\EventAcronym{ICCSW}
\EventYear{2017}
\EventDate{September 26th and 27th, 2017}
\EventLocation{Imperial College London, United Kingdom}
\EventLogo{}
\SeriesVolume{42}
\ArticleNo{5}
% Editor-only macros::end %%%%%%%%%%%%%%%%%%%%%%%%%%%%%%%%%%%%%%%%%%%%%%%


% Page numbers in side margins
\usepackage[some]{background}
\usepackage{ifthen}
\SetBgContents{\textcolor{gray}{\LARGE \bf \hl \thepage}}
\SetBgOpacity{1}
\SetBgScale{1}
\SetBgAngle{0}
\SetBgColor{black}
\makeatletter
  \AddEverypageHook{%
    \ifthenelse{\isodd{\thepage}}%
      {\SetBgPosition{.9\paperwidth,-.9\paperheight}%
	  \thispagestyle{empty}%
	}%
      {\SetBgPosition{.1\paperwidth,-.9\paperheight}%
	  \thispagestyle{empty}%
	}%
    \bg@material}
\makeatother
\begin{document}

\maketitle

\begin{abstract}
Linear discriminative models, in the form of Logistic Regression, are a popular choice within the clinical domain in the development of risk models. Logistic regression is commonly used as it offers explanatory information in addition to its predictive capabilities. In some examples the coefficients from these models have been used to determine overly simplified clinical risk scores. Such models are constrained to modeling linear relationships between the variables and the class despite it known that this relationship is not always linear. This paper compares the conditions under which linear discriminative and linear generative models perform best. This is done through comparing logistic regression and naïve Bayes on real clinical data. The work shows that generative models perform best when the internal representation of the data is closer to the true distribution of the data and when there is a very small difference between the means of the classes. When looking at variables such as sodium it is shown that logistic regression can not model the observed risk as it is non-linear in its nature, whereas naïve Bayes gives a better estimation of risk. The work concludes that the risk estimations derived from discriminative models such as logistic regression need to be considered in the wider context of the true risk observed within the dataset.
 \end{abstract}
\end{document}
