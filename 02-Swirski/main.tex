\documentclass[a4paper,UKenglish]{oasics-v2016}
\usepackage{microtype}
\bibliographystyle{plain}

\title{JavaScript: A Whirlwind tour}
\titlerunning{JavaScript: A Whirlwind tour}

\author{Leszek Swirski}
\affil{Google Inc.}
\authorrunning{Leszek Swirski}
\Copyright{Leszek Swirski}


\subjclass{D.2 Software Engineering}
\keywords{Javascript}

%Editor-only macros:: begin (do not touch as author)%%%%%%%%%%%%%%%%%%%%%%%%%%%%%%%%%%
\EventEditors{Fergus Leahy and Juliana Franco}
\EventNoEds{2}
\EventLongTitle{2017 Imperial College Computing Student Workshop (ICCSW 2017)}
\EventShortTitle{ICCSW 2017}
\EventAcronym{ICCSW}
\EventYear{2017}
\EventDate{September 26—-27, 2017}
\EventLocation{London, UK}
\EventLogo{../graphics/iccsw17}
\SeriesVolume{60}
\ArticleNo{02}
%%%%%%%%%%%%%%%%%%%%%%%%%%%%%

\begin{document}
\maketitle
\begin{abstract}
In this talk Chris DiBona will review Google’s use of open source projects and the history of prominent releases like Android, Chromium, Angular.js and some 3500 other projects (though not all of them will be surveyed!). 
Keeping such releases on track and efficient and, in some cases, retiring them has been his focus since he started at the company.
He’ll review the various ways Google supports the worldwide community of software developers that it has derived so much value from. 
Specifically for the students assembled, Mr. DiBona will also talk about the university oriented program “The Summer of Code” which is designed 
to lure students into open source projects and provide for them the real world mentorship they need to become open source committers themselves. 
A paid internship that lasts approximately 3 months during the summer months, The Summer of Code has introduced over 10,000 developers in 123
countries to open source software development and added over 30 million lines of code to open source 
projects that Google and the students use every day of their lives.
\end{abstract}
\end{document}
