\documentclass[a4paper,UKenglish]{oasics}
\usepackage{microtype}
\bibliographystyle{plain}

\title{Google and Open Source}
\titlerunning{Google and Open Source}

\author{Chris DiBona}
\affil{Google Inc.}
\authorrunning{Chris DiBona}
\Copyright{Chris DiBona}


\subjclass{D.2 Software Engineering}
\keywords{Open Source}

%Editor-only macros:: begin (do not touch as author)%%%%%%%%%%%%%%%%%%%%%%%%%%%%%%%%%%
\serieslogo{../graphics/iccsw15}%please provide filename (without suffix)
\volumeinfo%(easychair interface)
  {Claudia Schulz and Daniel Liew}% editors
  {2}% number of editors: 1, 2, ....
  {2015 Imperial College Computing Student Workshop (ICCSW 2015)}% event
  {49}% volume: please ask the editorial office before starting typesetting
  {1}% issue
  {1}% starting page number
\EventShortName{ICCSW'15}
\DOI{10.4230/OASIcs.ICCSW.2015.1} 
%%%%%%%%%%%%%%%%%%%%%%%%%%%%%

\begin{document}
\maketitle
\begin{abstract}
In this talk Chris DiBona will review Google’s use of open source projects and the history of prominent releases like Android, Chromium, Angular.js and some 3500 other projects (though not all of them will be surveyed!). 
Keeping such releases on track and efficient and, in some cases, retiring them has been his focus since he started at the company.
He’ll review the various ways Google supports the worldwide community of software developers that it has derived so much value from. 
Specifically for the students assembled, Mr. DiBona will also talk about the university oriented program “The Summer of Code” which is designed 
to lure students into open source projects and provide for them the real world mentorship they need to become open source committers themselves. 
A paid internship that lasts approximately 3 months during the summer months, The Summer of Code has introduced over 10,000 developers in 123
countries to open source software development and added over 30 million lines of code to open source 
projects that Google and the students use every day of their lives.
\end{abstract}
\end{document}
