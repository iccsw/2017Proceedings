\documentclass[a4paper,UKenglish]{oasics-v2016}
\usepackage{microtype, amsmath, stmaryrd, verbatim}
\bibliographystyle{plainurl}

\title{Equivalence of Probabilistic $\mu$-Calculus and p-Automata\footnote{A 
longer version of this paper appeared in the proceedings of CIAA 2017 
\cite{ciaa17}}}
\author[]{Claudia Cauli}	
\author[]{Nir Piterman}
\affil[]{University of Leicester, Department of Informatics, Leicester, UK\\
\texttt{cc488@le.ac.uk, nir.piterman@gmail.com}}

\authorrunning{C. Cauli and N. Piterman}
\Copyright{Claudia Cauli and Nir Piterman}

\subjclass{F.1.2. Models of Computation - Probabilistic Computation, F.4.1. 
Mathematical Logic, F.4.2. Formal Languages}
\keywords{$\mu$-Calculus, Automata, Probabilistic Models}

%Editor-only macros:: begin (do not touch as 
%author)%%%%%%%%%%%%%%%%%%%%%%%%%%%%%%%%%%
\EventEditors{Fergus Leahy and Juliana Franco}
\EventNoEds{2}
\EventLongTitle{2017 Imperial College Computing Student Workshop (ICCSW 2017)}
\EventShortTitle{ICCSW 2017}
\EventAcronym{ICCSW}
\EventYear{2017}
\EventDate{September 26—-27, 2017}
\EventLocation{London, UK}
\EventLogo{../graphics/iccsw17}
\SeriesVolume{60}
\ArticleNo{03}
% Editor-only macros::end %%%%%%%%%%%%%%%%%%%%%%%%%%%%%%%%%%%%%%%%%%%%%%%

\begin{document}
	
	\maketitle
	
	\begin{abstract}
An important characteristic of Kozen's $\mu$-calculus is its
strong connection with parity alternating tree automata.
Here, 
we show that the probabilistic $\mu$-calculus, $\mu^p$-calculus, and
p-automata (parity alternating Markov chain automata) have an
equally strong connection.
Namely, for every $\mu^p$-calculus formula we can construct a
p-automaton that accepts exactly those Markov chains that satisfy
the formula.
For every p-automaton we can construct a $\mu^p$-calculus formula
satisfied in exactly those Markov chains that are accepted by the
automaton.
The translation in one direction relies on a normal form of the
calculus and in the other direction on the usage of vectorial
$\mu^p$-calculus.
	\end{abstract}

\section{Introduction}
The verification of probabilistic systems is an increasingly important area
that has led to the development of new formalisms and tools for the evaluation
of quantitative properties over stochastic models.
These tools range from temporal logics and quantitative variants of Kozen's
modal $\mu$-calculus \cite{Koz83} to probabilistic automata and games
\cite{BK08,HK97,Mio2012a,Mio12,MS15}.

This work focuses on two such formalisms, $\mu^p$-calculus and p-automata.
The $\mu^p$-calculus has been introduced in \cite{CKP15} as a
probabilistic extension of Kozen's modal $\mu$-calculus.
The so-called p-automata \cite{CP13} are probabilistic alternating
parity automata that read Markov chains as input. 
Acceptance of a Markov chain by a p-automaton is decided by an
obligation game, that is, a turn-based stochastic parity game with
obligations. 

The key contribution given by this paper is the proof of the
equivalence between $\mu^p$-calculus and p-automata. 
We provide a framework to translate $\mu^p$-formulas into p-automata
and, using the vectorial syntax, define the inverse conversion from
p-automata into $\mu^p$-calculus. Thus, we show that the two formalisms
have the same expressive power and that they enjoy a 
close
relationship similar to those of Kozen's $\mu$-calculus and
alternating tree automata (see below).
%
\subparagraph{Related Work.}This study belongs to a general field of research 
that 
aims to define a connection between logics and automata theory.
An interesting survey conducted by Kupferman et al. in \cite{KVW00} provides 
insights into this relationship by presenting translations from a number of 
temporal logics -- linear-time, branching-time, $\mu$-calculus, and its 
alternation-free fragment -- into different classes of alternating tree 
automata.

Over the last three decades, several studies have focused on the definition of 
an automata-theoretic approach to Kozen's $\mu$-calculus.
In \cite{EJ91}, Emerson and Jutla proposed a framework to convert 
$\mu$-calculus formulas into alternating tree automata, then reduced to their 
non-deterministic counterpart.
Their result complements previous studies by Niwi\'{n}ski \cite{Ni88} that 
defined the inverse translation from non-deterministic tree automata to 
$\mu$-calculus, thus showing that Kozen's $\mu$-calculus is equivalent to 
tree automata in expressive power.
In \cite{JW95}, Janin and Walukiewicz introduced $\mu$-automata, alternating 
automata with a parity acceptance condition that easily translate to and from 
$\mu$-calculus formulas in disjunctive normal form.
In a subsequent work, \cite{Ni97}, Niwi\'{n}ski extends his previous result to 
a broader scope establishing the equivalence between alternating automata over 
arbitrary algebraic structures -- thus including trees -- and fixed point 
terms, a general fixpoint formalism that finds a natural interpretation as a 
system of equations.
Wilke, in \cite{Wil01}, addresses the interplay among $\mu$-calculus, 
alternating tree automata, and games.
In particular, he gives a translation from logic to automata and then defines 
the acceptance problem for automata by reduction to the winner problem in 
parity games.
A comprehensive outline of the relationship among logics, automata, and parity 
games
is given in \cite{GTW02}. Overviews of $\mu$-calculus, including its 
mathematical foundation, axiomatic system, properties, guarded form, vectorial 
syntax, game semantics, and equivalence with automata, can be found in 
\cite{AN01,BBW07,BS07,BW15}.

Huth and Kwiatkowska suggested a quantitative $\mu$-calculus to reason
about Markov chains \cite{HK97}.
This calculus was extended in \cite{CIN05} by adding a bounded number
of probabilistic quantifications and allowing to define PCTL*.
The $\mu^p$-calculus allows to nest probabilistic quantifications
inside fixpoint operators and, thus, allows for unbounded
probabilistic quantifications.
It is a subset of the calculus defined in \cite{Mio2012a,MS15}.
The $\mu^p$-calculus is 
expressive enough to include PCTL 
and PCTL*, the complexity of its decision procedures is reduced, and
the algorithms involved wrap up standard algorithms for solution of
(quantitative) two-player stochastic parity games in extra layer
rather than bespoke algorithms. 

%
%
%
\section{Background}
%
%
%
\subsection{Markov Chains}
A Markov chain $M$ over a set $AP$ of atomic propositions is a
probabilistic labelled transition system defined by the tuple $M=(S,
s^{in}, L, P)$, where $S$ is the set of locations; $s^{in} \in S$ is
the initial location; $L$ is a labelling function, overloaded to both
denote $L: S \rightarrow 2^{AP}$ and $L: AP \rightarrow 2^S$; and $P$
is the probability transition function $P: S \times S \rightarrow
[0,1]$. 
For every location $s$ we define the set $succ(s)$ of its successors as the set 
of locations $s^\prime$ such that $P(s,s^\prime)>0$.
Clearly, the sum of the probabilities of moving from a location to all its 
successors must be equal to $1$, that is $\textstyle\sum_{s^\prime \in succ(s)} 
P(s,s^\prime) = 1$, and every location has at least one successor.
%
%
\subsection{Obligation Games}
Obligation games \cite{CP13} are two-player stochastic parity games with 
obligations that are played on a graph amongst a probabilistic system and two 
players, called Player 0 and Player 1.
\begin{definition}[Obligation Game]
	An obligation game $G$ is the tuple $$G = \big((V, E),(V_0, V_1, 
	V_p),\mathcal{K}, \left\langle \alpha, O \right\rangle \big),$$ where $V$ 
	is 
	the set of configurations, partitioned in Player 0 ($V_0$), Player 1 
	($V_1$), 
	and probabilistic configurations ($V_p$); $E \subseteq V \times V$ is the 
	set 
	of edges; $\mathcal{K}: V_p\times V \rightarrow [0,1]$ is the probability 
	function such that $(v,v^\prime)\notin E$ implies
	$\mathcal{K}(v,v^\prime)=0$ and for every $v\in V_p$ we have
	$\textstyle\sum_{v^\prime\in V}\mathcal{K}(v,v^\prime)=1$;
	and the pair $\left\langle \alpha, O \right\rangle$ defines the 
	\emph{goal}: 
	$\alpha : V \rightarrow [0..k]$ is the parity condition, and $O : V 
	\rightarrow 
	(\lbrace >, \geq \rbrace \times [0,1]) \cup \lbrace \bot\rbrace $ marks the 
	obligations, with the symbol $\bot$ denoting no obligation.
\end{definition}
Obligations are statements applied to some configurations that impose 
constraints on the winning paths that depart from them.
An obligation has the form $>\!x$ or $\geq\!x$, where $x \in [0,1]$, so as to 
indicate that the measure of the paths starting from that configuration must be 
greater than, or greater than or equal to, a given value $x$.
Fixing a pair of strategies -- $\sigma$ for Player 0 and $\pi$ for Player 1 -- 
and a prefix of configurations $w\in V^+$, the game is reduced to only 
probabilistic vertices and, hence, can be seen as a Markov chain enriched with 
a \emph{goal} $\langle \alpha,O\rangle$, that is, a winning condition and a set 
of obligations.
We denote such structure as $G^{w(\sigma,\pi)}$.
Value and winner of $G$ are decided by analysing $G^{w(\sigma,\pi)}$ using the 
notion of \emph{choice set}.

A choice set is the set of finite paths that extend the prefix $w$ and end in 
a configuration with an obligation that can be met.
It must be extended through infinite paths that either reach another obligation 
or never reach another obligation and are winning.
The measure of the choice set is the measure of these infinite paths and 
determines the value of the game on every configuration for each player, 
denoted as $val_i(v)$ for $i\in\lbrace 0,1\rbrace$.
We refer the reader to \cite{CP13} for further details concerning the measure 
of choice sets and the value of obligation parity Markov chains.
We write the value of game $G$ on configuration $v$ as $val_G(v)$ and we define 
it as the value for Player 0 on $v$. 

Player 0 wins the game $G$ from prefix $w$ with a value of $1$ if for
every value $r<1$ there exists a strategy $\sigma$ such
that for all Player 1's strategies $\pi$ in the corresponding Markov
chain $G^{w(\sigma, \pi)}$ it is possible to determine a choice set
whose measure is at least $r$.  
%
%
\subsection{The $\mu^p$-Calculus}
The $\mu^{p}$-calculus \cite{CKP15} is an extension of Kozen's 
$\mu$-calculus \cite{Koz83} that allows one to specify properties that are 
bounded by a specific probability.
This is done through the distinction between qualitative ($\Phi$) and 
quantitative ($\Psi$) formulas that are evaluated to values in the sets 
$\lbrace 0,1 \rbrace$ and $[0,1]$, respectively.
A $\mu^p$-calculus sentence is qualitative and might contain one or more 
quantitative sub-formulas within a probabilistic quantification operator 
$[\cdot]_J$.
The operator $[\cdot]_J$ checks whether the value of the enclosed formula 
satisfies the bound $J$ and gets the value $1$ or $0$ accordingly.
The syntax of the $\mu^p$-calculus is given by the following BNF 
grammar.\vspace*{5pt}\\
%%%%% SYNTAX %%%
\hspace*{15pt}$J ::=\ \lbrace \geq, > \rbrace \times [0,1]$

$\Phi ::=\ p \mid \neg p \mid \varphi_1 \wedge \varphi_2 \mid \varphi_1 \vee 
\varphi_2 \mid [\Psi]_J \mid \mu X_i.\varphi \mid \nu X_i.\varphi $

$\Psi ::=\ \Phi \mid X_i \mid \psi_1 \wedge \psi_2 \mid \psi_1 \vee \psi_2 \mid 
\bigcirc \psi \mid \mu X_i.\psi \mid \nu X_i.\psi$\vspace*{5pt}\\
Fixed point formulas are of the form $\sigma X_i.\varphi$, where $\sigma \in 
\lbrace \mu,\nu \rbrace$ and $X_i$ is a variable in the set 
$\mathcal{V}=\lbrace X_0, X_1, ... \rbrace$.
Variable $X_i$ is bound by the fixed point operator to the formula $\varphi$, 
which we also denote by $\varphi(X_i)$ or $\varphi_{X_i}$.
%%%%% SEMANTICS %%%%%
\subparagraph{Semantics.}The semantics of a $\mu^p$-calculus formula $\varphi$ 
is 
given with respect to a Markov chain $M$ and an interpretation $\rho: 
\mathcal{V} \rightarrow (S \rightarrow [0,1])$.
That is, $\rho$ associates a function from locations to values in the domain 
$[0,1]$ with each variable $X_i$ appearing in $\varphi$.
Therefore, the semantics is a mapping of type $S \rightarrow [0,1]$ denoted by 
$\llbracket \varphi \rrbracket^\rho_M$ and defined as follows:
$$
\begin{array}{r c l r c l}
\llbracket p \rrbracket^{\rho}_{M}  & = & \chi_{L(p)} &
\llbracket \neg p \rrbracket^{\rho}_{M}  & = &  1- \chi_{L(p)} \\[3pt]
\llbracket \varphi_{1} \wedge \varphi_{2} \rrbracket^{\rho}_{M} & = & \min 
\lbrace 
\llbracket \varphi_{1} \rrbracket^{\rho}_{M},\llbracket \varphi_{2} 
\rrbracket^{\rho}_{M} \rbrace & 
\llbracket \varphi_{1} \vee \varphi_{2} \rrbracket^{\rho}_{M} & = & \max 
\lbrace 
\llbracket \varphi_{1} \rrbracket^{\rho}_{M},\llbracket \varphi_{2} 
\rrbracket^{\rho}_{M} \rbrace \\[3pt]
\llbracket X \rrbracket^{\rho}_{M} & = & \rho(X) &
\multirow{2}{*}{\mbox{$\llbracket [\varphi]_{J} \rrbracket^{\rho}_{M} $}} &
\multirow{2}{*}{ = } &
\multirow{2}{*}{
	\mbox{$
		\left \{
		\begin{array}{l l}
		1 & \mbox{If } \llbracket \varphi \rrbracket^{\rho}_{M} (s)J \\[3pt]
		0 & \mbox{Otherwise}
		\end{array} \right .
		$}
}
\\[3pt]
\llbracket \bigcirc \varphi \rrbracket^{\rho}_{M} & = & \lambda s. 
\textstyle\sum_{s^{\prime}} P(s,s^{\prime}) \llbracket \varphi 
\rrbracket^{\rho}_{M}(s^{\prime})  \\[3pt]
\llbracket \mu X.\varphi \rrbracket^{\rho}_{M} & = & \text{lfp}(\lambda 
f.\llbracket \varphi 
\rrbracket^{\rho[f/X]}_{M}) & 
\llbracket \nu X.\varphi \rrbracket^{\rho}_{M} & = & \text{gfp}(\lambda 
f.\llbracket \varphi 
\rrbracket^{\rho[f/X]}_{M})
\end{array}
$$    
Where $\chi_{L(p)}$ is the function that associates to a location $s$ the value 
$0$ if $s\notin L(p)$, or the value $1$ if $s\in L(p)$.
The only elements of the calculus that are evaluated exclusively to values in 
the set $\lbrace 0,1 \rbrace$ are $p$, $\neg p$, and $[\cdot]_{J}$.
All the other operators get real values in $[0,1]$, thus, can specify both 
quantitative and qualitative properties depending on their nested sub-formulas.

The alternation depth of a formula 
$\varphi$, denoted by $ad(\varphi)$, is the maximum number of fixed point 
operators that occur nested and alternated \cite{EL86,CKP15}.
In addition to the alternation depth, with every $\mu^p$-calculus sub-formula 
$\psi$ of $\varphi$ is associated a colour $c(\psi)$. If $\psi$ is a greatest 
fixed point then $c(\psi)=2\big(ad(\varphi)-ad(\psi)\big)$; if $\psi$ is a 
least fixed point then $c(\psi)=2\big(ad(\varphi)-ad(\psi)\big)+1$; and in 
every other case $c(\psi)=2ad(\varphi)-1$.
%
%
%
\subparagraph{Game Semantics \cite{CKP15}.} Given a $\mu^p$-calculus formula 
$\varphi$ 
and 
a Markov chain $M$, we construct an obligation game $G_{M,\varphi}$ and we 
refer to such game as \emph{semantics game}. The value on 
configuration $(s, \varphi)$ equals the semantics of the $\mu^p$-formula 
$\varphi$ over the location $s$ of the Markov chain. We say that $M$ 
\textit{satisfies} $\varphi$, denoted $M \models \varphi$ iff $\llbracket 
\varphi \rrbracket ^\rho_M (s^{in})=1$, which in terms of game semantics 
corresponds to $M \models \varphi$ iff $val_{G_{M,\varphi}}(s^{in},\varphi)=1$.
%
%
%
\begin{comment}
\subsubsection*{Game Semantics.}Given a $\mu^p$-calculus formula $\varphi$ and 
a Markov chain $M$, we construct an obligation game $G_{M,\varphi}$ and we 
refer to such game as \emph{semantics game}. Game $G_{M,\varphi}$ is the tuple 
$\big( (V,E),(V_0,V_1,V_p),\mathcal{K},\langle\alpha,O \big\rangle)$, where:
\begin{itemize}
	\item $V = S\times sub(\varphi)$
	\item $V_0 = \lbrace (s, \varphi_1 \vee \varphi_2) \mid 
	\varphi_1\vee\varphi_2 
	\in sub(\varphi) \rbrace$
	\item $V_1 = \lbrace (s, \varphi_1 \wedge \varphi_2) \mid 
	\varphi_1\wedge\varphi_2 \in sub(\varphi) \rbrace$
	\item $V_p = V \setminus (V_0 \cup V_1)$
	\item $E = \lbrace \big((s,p),(s,p)\big),\big((s,\neg p),(s,\neg p)\big) 
	\rbrace
	\cup \lbrace \big((s,X),(s,\sigma X. \varphi(X))\big) \rbrace\ \cup$\\
	\hspace*{21pt}$\lbrace \big((s,\varphi_1 \vee \varphi_2),(s,\varphi_i)\big) 
	\mid i \in \lbrace 1,2 \rbrace \rbrace \cup\ \lbrace \big((s,\varphi_1 
	\wedge 
	\varphi_2),(s,\varphi_i)\big) \mid i \in \lbrace 1,2 \rbrace \rbrace\ 
	\cup$\\
	\hspace*{21pt}$\lbrace \big((s,\bigcirc \varphi),(s^\prime,\varphi)\big) 
	\mid 
	s^\prime \in succ(s) \rbrace\ \cup \lbrace 
	\big((s,[\varphi]_J),(s,\varphi)\big) \rbrace\ \cup$\\
	\hspace*{21pt}$\lbrace \big((s,\sigma X. \varphi(X)),(s,\varphi(X))\big) 
	\mid 
	\sigma \in \lbrace \mu,\nu \rbrace \rbrace$
	\item $\mathcal{K}((s,\bigcirc \psi),(s^\prime, \psi))=P(s,s^\prime)$
	\item $\alpha(s,\psi) = \begin{cases}0&\mbox{ if }\psi=p, p\in L(s)\mbox{ 
	or 
	}\psi=\neg p, p\notin L(s)\\
	1&\mbox{ if }\psi=p, p\notin L(s)\mbox{ or }\psi=\neg p, p\in L(s)\\
	c(\psi) &\mbox{ otherwise}\end{cases}$
	\item $O(s,[\psi]_J) = J,\ \ O(v)=\bot \mbox{ for all other }v\in V$
\end{itemize}
\begin{lemma}{\emph{\cite{CKP15}}}
	For every Markov chain $M$, every location $s$, and every formula $\varphi$ 
	we 
	have $\llbracket\varphi\rrbracket^\rho_M (s)= val_0(s,\varphi)$, where 
	$val_0(s,\varphi)$ is the value of configuration $(s,\varphi)$ in game 
	$G_{M,\varphi}$.
\end{lemma}
For a qualitative $\mu^p$-calculus formula $\varphi$ we say that $M$ satisfies 
$\varphi$, denoted $M, s^{in} \models\varphi$ or $M\models\varphi$, iff 
$\llbracket \varphi\rrbracket^\rho_M(s^{in}) = 1$. That is, $M\models\varphi 
\mbox{ iff }val_{G_{M,\varphi}}(s^{in}, \varphi)=1$.
\end{comment}
%
%
\subsection{p-Automata}
A p-automaton $A$ is an alternating parity automaton that reads Markov chains 
as input \cite{CP13}.
From a state $q$, the p-automaton reads a location $s$ of a Markov chain $M$ 
and, according to the labelling $L(s)$, performs a transition.
Since Markov chains have probabilities and the paths starting from a location 
$s$ are characterised by a measure, the p-automaton $A$ might need to mark the 
states by a bound $J$.
The bound $J$ is an element of the set $(\lbrace \geq,> \rbrace \times [0,1])$ 
and, analogously to the obligations over configurations of games, imposes a 
constraint over the measure of the accepted paths starting in $s$.

For the set of states $Q$, we denote by $\llbracket Q \rrbracket_{>}$ the set 
of states $q \in Q$ that are marked by a bound $J$ defined as $\lbrace 
\llbracket q \rrbracket_{J} \mid q\!\in\!Q,\ J\!\in\!(\lbrace \geq,>\rbrace 
\times [0,1]) \rbrace$. Moreover, we denote by $B^{+}(X)$ the set of 
\emph{positive boolean formulas} over elements $x$ in the set $X$, given by the 
following grammar:
$$ \theta ::= x \mid true \mid false \mid \theta_{1}\wedge\theta_{2} \mid 
\theta_{1}\vee\theta_{2}$$ 
Given a formula $\theta \in B^{+}(X)$ its \emph{closure} $\textsf{cl}(\theta)$ 
is the set of all sub-formulas of $\theta$ defined according to the grammar 
above.
For a set $\Theta$ of formulas, the closure is computed as 
$\textsf{cl}(\Theta)=\bigcup_{\theta \in \Theta}\textsf{cl}(\theta)$.
\begin{definition}[p-Automata]
	A p-automaton $A$ over the set $AP$ of atomic propositions is defined by 
	the 
	tuple: 
	$$A = (\Sigma, Q, \varphi^{in}, \delta, \Omega)$$
	where $\Sigma=2^{AP}$ is a finite input alphabet, $Q$ is a possibly 
	infinite 
	set of states, $\varphi^{in}\in B^+(\llbracket Q\rrbracket_>)$ is the 
	initial 
	condition, $\delta:Q\times\Sigma\rightarrow B^+(Q \cup \llbracket 
	Q\rrbracket_>)$ is the transition function, and 
	$\Omega:Q\rightarrow[0\hdots 
	k]$ is the parity acceptance condition.
\end{definition}
%
%
\subparagraph{Acceptance Game \cite{CP13}.}The set of Markov chains accepted by 
a 
p-automaton $A$ is the language of $A$, denoted by $\mathcal{L}(A)$.
Acceptance of a Markov chain $M$ by $A$ is decided through the obligation game 
$G_{M,A}$. The Markov chain $M$ is accepted if the configuration 
$(s^{in},\varphi^{in})$ 
has value 1 in $G_{M,A}$. That is, $M\in\mathcal{L}(A)\mbox{ iff 
}val_{G_{M,A}}(s^{in}, \varphi^{in})=1$.
%
\begin{comment}
\subsubsection*{Acceptance Game} The set of Markov chains accepted by a 
p-automaton $A$ is the language of $A$, denoted by $\mathcal{L}(A)$.
Acceptance of a Markov chain $M$ by $A$ is decided through the obligation game 
$G_{M,A} = (V,E,(V_0,V_1,V_p),\mathcal{K}, \mathcal{G})$, where:
\begin{itemize}
	\item $V = S \times \textsf{cl} (\delta(Q,\Sigma))$
	\item $V_{0} = \lbrace (s, \theta_{1} \vee \theta_{2}) \mid s\in S \text{ 
	and 
	}\theta_{1} \vee \theta_{2} \in \textsf{cl} (\delta(Q,\Sigma))\rbrace$
	\item $V_{1} = \lbrace (s, \theta_{1} \wedge \theta_{2}) \mid s\in S \text{ 
	and 
	}\theta_{1} \wedge \theta_{2} \in \textsf{cl} (\delta(Q,\Sigma))\rbrace$
	\item $V_{p} = V \setminus (V_{0} \cup V_{1})$
	\item $E = \lbrace \big((s,\theta_1 \wedge \theta_2),(s,\theta_i)\big) \mid 
	i 
	\in \lbrace 1,2 \rbrace \rbrace \ \cup$\\
	\hspace*{21pt}$\lbrace \big((s,\theta_1 \vee \theta_2),(s,\theta_i)\big) 
	\mid i 
	\in \lbrace 1,2 \rbrace \rbrace \ \cup $\\
	\hspace*{21pt}$\lbrace \big((s,q),(s^{\prime},\delta(q,L(s)))\big) \mid 
	s^{\prime}\in succ(s)\rbrace \ \cup\ $\\
	\hspace*{21pt}$\lbrace \big((s,\llbracket q 
	\rrbracket_{J}),(s^{\prime},\delta(q,L(s)))\big) \mid s^{\prime}\in succ(s) 
	\rbrace$
	\item $\mathcal{K}\big((s,q),(s^{\prime},\delta(q,L(s)))\big)\ =\ 
	\mathcal{K}\big((s,\llbracket q 
	\rrbracket_{J}),(s^{\prime},\delta(q,L(s)))\big)\ =\ P(s,s^{\prime})$
	\item $\mathcal{G} = \langle \alpha, O \rangle$, where $\alpha(s,q) = 
	\alpha(s,\llbracket q \rrbracket_J) = \Omega(q)$, and $O(s,\llbracket q 
	\rrbracket_J)= J$.
\end{itemize}
The Markov chain $M$ is accepted if the configuration $(s^{in},\varphi^{in})$ 
has value 1 in $G_{M,A}$. That is, $M\in\mathcal{L}(A)\mbox{ iff 
}val_{G_{M,A}}(s^{in}, \varphi^{in})=1$.
\end{comment}
\section{Vectorial $\mu^p$-Calculus}
We introduce the vectorial form as an alternative syntax for formulas in 
$\mu^p$-calculus.
This form exposes the distinction between the fixpoint operators, which appear 
as a prefix of the formula, and the modal formulas that they bind, allowing one 
to focus on the modal properties rather than on an intricate nesting of 
fixed-point terms.
Through this syntax, the alternation depth of a sentence is easier to identify, 
as the number of pairwise distinct fixpoint operators within the prefix of the 
formula, and the most complex properties can be expressed in a succinct way.

Let $F_i$ be the set of functions $(S \rightarrow [0,1])_i$ from locations to 
values in the unit interval, and $\varphi_i$ be a modal $\mu^p$-formula over 
the product lattice $(F_1 \times\hdots\times F_n)^m$ with range $F_i$, i.e. 
$\varphi_i$ takes $m$ vectors of $n$ variables $\langle 
X_1^1,\hdots,X_n^1,\hdots,X_1^m,$ $\hdots,X_n^m \rangle$ and evaluates to a 
single function in $F_i$.
If we consider the vector $\vec\varphi$ of all modal terms 
$\langle\varphi_1,\hdots,\varphi_n\rangle$, each of which has range $F_i$, 
then, $\vec\varphi$ can be seen as a mapping of type 
$\vec\varphi:(F_1\times\hdots\times F_n)^m\rightarrow F_1\times\hdots\times 
F_n$, whose monotonicity derives from the monotonicity of each single component 
and for which, by the Knaster-Tarski theorem, least and greatest fixpoints are 
always defined.
For $m=1$, we denote as $\mu\vec{X}.\vec\varphi$, resp. 
$\nu\vec{X}.\vec\varphi$, the least, resp. greatest, fixpoint of the mapping 
$\vec\varphi$, as a compact notation for:
\vspace*{5pt}\\
\hspace*{100pt}	$
\sigma
\begin{pmatrix}
X_1\\
\vdots\\
X_n
\end{pmatrix}
.
\begin{pmatrix}
\ \varphi_1(X_1,\hdots,X_n)\ \\
\ \vdots\ \\
\ \varphi_n(X_1,\hdots,X_n)\ 
\ \end{pmatrix}
=
\begin{pmatrix}
f_1\\
\vdots\\
f_n
\end{pmatrix}.
$
\vspace*{5pt}\\
Vectorial $\mu^p$-calculus has the same expressive power as scalar 
$\mu^p$-calculus.
By the application of the Beki\v{c} principle \cite{AN01}, whose effect is to 
push the fixpoint operators inwards, every vectorial formula $\sigma \vec{X}. 
\vec{\varphi}$ can be reduced to a vector $\vec{f}$ of scalar formulas $\langle 
f_1,\hdots,f_n\rangle$. 
\subparagraph*{Semantics.}The semantics of a vectorial formula 
$\sigma\vec{X}.\vec\varphi$ is the vector of 
semantics: 
$\llbracket\sigma\vec{X}.\vec\varphi\rrbracket^\rho_M=\llbracket\sigma 
X_1.\vec\varphi\rrbracket^\rho_M\times\hdots\times\llbracket\sigma 
X_n.\vec\varphi\rrbracket^\rho_M.$
We use the projection operator on vectors $\downarrow_i$ to select the $i$-th 
component: $\llbracket\sigma\vec{X}.\vec\varphi\downarrow_i\rrbracket^\rho_M = 
\llbracket \sigma X_i.\vec\varphi\rrbracket^\rho_M = f_i$. 
The meaning of choosing a component is to define an \emph{entry point} to the 
computation performed by the vectorial formula \cite{BFL15}.
\subparagraph{Game Semantics.} As for standard $\mu^p$-calculus, the semantics 
of a vectorial sentence can be defined by an obligation game 
$G_{M,\vec\varphi}$ that results from the composition of sub-games 
for each of the $n$-th components of the vectorial formula. 
The value of such game is a vector of values, each corresponding to 
$val_{G_{M,\vec\varphi}}(s, \varphi_i)$.
%
%
\begin{comment}
\subparagraph{Game Semantics.}
The semantics of a vectorial $\mu^p$-calculus sentence 
$\sigma_1\vec{X}_1\hdots$ $\sigma_m\vec{X}_m.\vec\varphi$ of depth $m$ and 
height $n$ for a Markov chain $M$ is given by the obligation game 
$G_{M,\vec\varphi}$ defined by the tuple 
$(V,E,(V_0,V_1,V_p),\mathcal{K},\langle \alpha,O\rangle)$.
The set of configurations of the game is the set of pairs of a location from 
the Markov chain and a subformula in $\bigcup_{i\leq n} sub(\varphi_i)$ and, 
since $\vec\varphi$ is a vector of modal formulas, do not contain pairs whose 
second element is a fixpoint term.
As a consequence of the absence of such configurations, vertices of the form 
$(s, X_i^j)$ link directly to $(s, \varphi_i)$ and carry the relevant priority 
$j$, which is the depth of the fixpoint that the variable $X_i^j$ binds in the 
vectorial formula.
The remaining components are defined exactly as in the semantics game for the 
scalar $\mu^p$-calculus. 

The value of the game $G_{M,\vec\varphi}$ on the initial location $s^{in}$ of a 
Markov chain $M$ is the vector of values: $val_{G_{M,\vec\varphi}}\ =\ 
val_{G_{M,\vec\varphi}}(s^{in},\varphi_1) \times\hdots\times 
val_{G_{M,\vec\varphi}}(s^{in},\varphi_n),$ where the value of the $i$-th 
component is $val_{G_{M,\vec\varphi}}(s^{in}, \varphi_i)$. 
\begin{lemma}\label{lem:semVectorial}For every Markov chain $M$, every location 
	$s$, every $\mu^p$-calculus vectorial sentence $\sigma_1\vec{X}_1\hdots$ 
	$\sigma_m\vec{X}_m.\vec\varphi$ of height $n$, and index $i\leq n$ we have 
	$$
	\llbracket\sigma_1\vec{X}_1\hdots\sigma_m\vec{X}_m.\vec\varphi\downarrow_i\rrbracket^\rho_M(s)
	= val_{G_{M,\vec\varphi}}(s, \varphi_i).
	$$
\end{lemma}
\begin{proof}
	The proof is conducted as that of Theorem 6 in \cite{CKP15} with the 
	exception 
	that configurations $(s,\sigma X.\varphi(X))$ do not appear in the game and 
	those of the form $(s,X_i)$ link directly to $(s,\varphi_i)$.
	\end{proof}
	
\end{comment}

\section{From $\mu^p$-Calculus to p-Automata}
We show that every qualitative $\mu^p$-calculus formula can be translated into 
an equivalent p-automaton.
The translation relies on the formulas satisfying some syntactic requirements.  
\subparagraph{Well-Formedness}
The set of well-formed $\mu^p$-calculus formulas is semantically equivalent to 
the standard form of the calculus; however, it poses some constraints on the 
syntax allowing for the conversion into p-automata.
We require that the variables be bound exactly once and that well-formed 
formulas be \emph{guarded}; that is, all the occurrences of a variable must be 
in the scope of a next modality, which is itself in the scope of a fixpoint 
operator.
To this end, formulas can be re-written in guarded form, as explained in 
\cite{KVW00,BW15}, by the iterated replacement of every open occurrence of a 
variable $X$ by \emph{false} in least fixed point formulas and by \emph{true} 
in greatest fixed point formulas.
Also, we consider the probabilistic quantification operator over a bound $J$ 
that is restricted to the set $(\lbrace \geq, > \rbrace\times [0,1])\setminus 
\lbrace \geq 0, >1 \rbrace $; this restriction does not affect the expressive 
power of the language since properties of the form $[\cdot]_{\geq 0}$ and 
$[\cdot]_{>1}$ correspond to \emph{true} and \emph{false} statements.
Moreover, we are interested in formulas where all the instances of the 
probabilistic quantification operator $[\cdot]_J$ are directly applied to a 
next $\bigcirc$.
This requirement is necessary because the statements enclosed in a 
probabilistic operator will translate into states of the corresponding 
automaton that performs a transition moving to read the next locations of the 
model.
One can achieve this form by transforming the formulas according to the 
equivalences stated in the lemma below.
\begin{lemma}The following $\mu^{p}$-calculus formulas are semantically 
	equivalent.
\begin{center}
		$\left[p\right]_J \equiv p$\\
	$\left[\neg p\right]_J \equiv \neg p $\\
	$\left[\varphi_1 \wedge \varphi_2\right]_J \equiv \left[\varphi_1\right]_J 
	\wedge \left[\varphi_2\right]_J$\\
	$\left[\varphi_1 \vee \varphi_2\right]_J \equiv \left[\varphi_1\right]_J 
	\vee 
	\left[\varphi_2\right]_J $\\
	$\left[\sigma X.\varphi(X)\right]_J \equiv \left[\varphi \big(\sigma 
	X.\varphi(X) 
	\big)\right]_J$
\end{center}
\end{lemma}
\begin{proof}
	The proof arises from the semantics of the $\mu^p$-calculus and the fixed 
	point 
	axioms. 
	\end{proof}
\subparagraph{Translation} Let $\varphi$ be a qualitative well-formed 
$\mu^p$-calculus formula over the set $AP$ of atomic propositions.
The p-automaton $A_\varphi$ is the tuple $(2^{AP}, Q, \delta, \varphi^{in}, 
\Omega)$, where $2^{AP}$ is the alphabet, $Q$ is the set of states 
$\lbrace\bot,\top\rbrace\cup\lbrace p,\neg p, (\bigcirc \psi, c) \mid$ $\mbox{ 
	for all }$ $p,\neg p,$ $\bigcirc\psi$ $\in sub(\varphi) \mbox{ and }c\in 
[0\hdots 2ad(\varphi)-1] \rbrace$, the transition function $\delta$
(and the auxiliary function $\delta_\epsilon$) is defined by the rules
in Figure~\ref{fig:transition of a_varphi}. 

\begin{figure}[bt]
	\begin{minipage}[t]{0.38\textwidth}
		\begin{align*}
		\delta(p,a)&=\begin{cases} 	\top&\mbox{if }p\in a \\ 
		\bot&\mbox{if }p\notin a \end{cases}\\	
		\delta(\neg p,a)&=\begin{cases} 	\top&\mbox{if }p\notin a \\ 
		\bot&\mbox{if }p\in a \end{cases}\\
		\delta(\top,a)&=\top\\
		\delta(\bot,a)&=\bot\\
		\delta\big((\bigcirc\psi, c),a\big)&=\delta_\epsilon(\psi,c)
		\end{align*}
	\end{minipage}
	\begin{minipage}[t]{0.57\textwidth}
		\begin{align*}
		\delta_\epsilon(p,c)&=p\\ 
		\delta_\epsilon(\neg p, c)&=\neg p\\
		\delta_\epsilon(\psi_1 \vee \psi_2, c)&=\delta_\epsilon(\psi_1,c) \vee 
		\delta_\epsilon(\psi_2, c)\\
		\delta_\epsilon(\psi_1 \wedge \psi_2, c)&=\delta_\epsilon(\psi_1,c) 
		\wedge 
		\delta_\epsilon(\psi_2, c)\\
		\delta_\epsilon\big(\sigma 
		X.\varphi(X),c\big)&=\delta_\epsilon\big(\varphi(X),c\big)\\
		\delta_\epsilon\big(X,c\big)&=\delta_\epsilon\big(\varphi(X), 
		c(X)\big)\\
		\delta_\epsilon(\bigcirc\psi, c)&=(\bigcirc\psi, c) \\
		\delta_\epsilon([\bigcirc\psi]_J,c)&=\llbracket (\bigcirc\psi, c) 
		\rrbracket_J
		\end{align*}
	\end{minipage}
	\caption{\label{fig:transition of a_varphi}Transition of $A_\varphi$.}
\end{figure}
The initial condition $\varphi^{in}$ is the expression 
$\delta_\epsilon\big(\varphi, c(\varphi)\big)$; the priority $\Omega$ is 
$\Omega(\bot) = 1$, $\Omega(\top)=0$, $\Omega(\bigcirc\psi,c)=c$, and maximum 
colour otherwise.

Transitions of $A_\varphi$ always consume the input label $a$ of a 
location in a Markov chain and move forward to its successors.
The computation starts from the states within the initial condition 
$\varphi^{in}$.
From a state $p$ or $\neg p$, the p-automaton reads the current label $a$ and 
moves to one of the special states $\top$ or $\bot$ defining an infinite 
computation that is accepting or rejecting, respectively.
When in a state $(\bigcirc \psi, c)$ reading a label $a$, the p-automaton moves 
to a new set of states determined by unfolding the formula $\psi$ through the 
epsilon transition function $\delta_\epsilon$.
The outcome of $\delta_\epsilon$, as well as of $\delta$, is a positive boolean 
formula over states $q$ and bounded states $\llbracket q \rrbracket_J$ that 
represents the requirement from the system.
States within such formula are evaluated over the successor locations in $M$.
Acceptance of a Markov chain $M$ by the p-automaton $A_\varphi$ is decided by 
the \emph{acceptance game} $G_{M,A_\varphi}$: if the value in such game of the 
initial configuration is 1 $M$ is accepted, otherwise, it is rejected. 
%

The following theorem states the correctness of the translation 
from $\mu^p$-calculus into p-automata.
\begin{theorem}\label{the:MupToPaut}
	Let $\varphi$ be a well-formed $\mu^p$-calculus formula and $A_\varphi$ the 
	automaton resulting from its translation. 
	Then, $\varphi$ and $A_\varphi$ are equivalent: the set of Markov chains 
	that 
	satisfy the formula $\varphi$ corresponds to the language 
	$\mathcal{L}(A_\varphi)$ recognised by the p-automaton $A_\varphi$.
	That is, 
	$M\!\models\!\varphi$ \emph{iff} $M\!\in\!\mathcal{L}(A_\varphi)$.
\end{theorem}
The proof is conducted by showing that for all Markov chains $M$ the 
\emph{acceptance game} $G_{M,A_\varphi}$ simulates the \emph{semantics game} 
$G_{M, \varphi}$.
In particular, there is a mapping between prefixes of paths in the two games, 
within which probabilities, obligations, and infinite winning sets are 
preserved.
Therefore, the acceptance game has the same value as that of the semantics 
game, leading the p-automaton $A_\varphi$ to accept all the Markov chains that 
satisfy the formula $\varphi$.

\section{From p-Automata to $\mu^p$-Calculus}
We show that every p-automaton can be translated into an equivalent 
$\mu^p$-calculus formula.
Transitions of p-automata define an infinite computation tree whose nodes are 
states marked by priorities.
The sequence of such priorities within the paths of the tree determines whether 
the computation is accepted or not: infinitely many visits to a minimal even 
priority mean acceptance, whereas passing infinitely often through a minimal 
odd priority causes rejection.
All these elements have their analogue in $\mu^p$-calculus: transitions and 
modal formulas, applying a transition from a state and passing through 
variables, odd/even priorities and least/greatest fixpoints, and levels of 
priorities and nesting of fixpoint formulas.
We exploit this analogy in the conversion from p-automata to $\mu^p$-calculus, 
using the syntax that most emphasises the role of each component, the vectorial 
form.
%
\subparagraph{Translation}Let $A$ be a p-automaton over the set $AP$ of atomic 
propositions defined by the tuple $(2^{AP}, Q, \delta, \varphi^{in}, \Omega)$, 
with $n$ the number of states of the automaton.  
Let $i_1,\hdots,i_m$ be the ordered chain of increasing priorities in the set 
$\bigcup_{q\in Q}\Omega(q)$.
For each index $j\leq m$ we introduce a vector $\vec{X}_{i_j}$ of $n+1$ fresh 
variables.
The first variable of each $j$-th vector is a dummy variable that refers to the 
initial condition of the automaton, and we indicate it as $X^{in}_{i_j}$.
The remaining $n$ variables of each $j$-th vector bind the $n$ formulas 
corresponding to the transitions that the p-automaton $A$ performs from each of 
its $n$ states; we write such variables as $X^1_{i_j},\hdots,X^n_{i_j}$.
Accounting for the initial condition of the automaton as the first component of 
these vectors allows one to retrieve the semantics of the resulting formula as 
the semantics of its first element.
As a consequence, the ordering of the other $n$ components is not relevant.
In order to turn states into variables we use the following function $t$:
\begin{center}

$t(\theta_1 \vee \theta_2) \ =\ t(\theta_1) \vee t(\theta_2)$\\
$t(\theta_1 \wedge \theta_2) \ =\ t(\theta_1) \wedge t(\theta_2)$\\
$t(\llbracket q \rrbracket_J) =\ [X^q_{\Omega(q)}]_J$\\
$t(q) \ =\ X^q_{\Omega(q)}$\\
\end{center}
We employ this function in the definition of the vector $\vec\varphi$ of modal 
$\mu^p$-formulas.
The first component of $\vec\varphi$ is $t(\varphi^{in})$, the other $n$ 
components are denoted by $\varphi^k$ for $k\leq n$ and are specified by the 
following modal formula
$$\varphi^k\ =\ \bigvee\limits_{a \in 2^{AP}}\left( \bigcirc\ t 
\left(\delta\left(q^k,a\right)\right) \wedge \bigwedge\limits_{p\in a}p\ \wedge 
\bigwedge_{p\notin a} \neg p\right).$$
Finally, the vectorial $\mu^p$-calculus formula $\vec\varphi_A$ is defined as 
the prefix 
chain of ordered fixpoints and vectors enclosing $\vec\varphi$, 
where $\sigma_{i_j}=\mu$ if $i_j$ is odd or $\sigma_{i_j}=\nu$ 
if $i_j$ is even:
$$
\vec{\varphi}_A\ =\ 
\sigma_{i_1} 
\begin{pmatrix}
X^{in}_{i_1}\\[0.3em]
X^{1}_{i_1}\\%[0.3em]
\vdots\\%[0.3em]
X^{n}_{i_1}\\
\end{pmatrix}
\hdots
\sigma_{i_m} 
\begin{pmatrix}
X^{in}_{i_m}\\[0.3em]
X^{1}_{i_m}\\%[0.3em]
\vdots\\%[0.3em]
X^{n}_{i_m}\\
\end{pmatrix}
.
\begin{pmatrix}
\ t\left(\varphi^{in}\right)\ \\[0.3em]
\varphi^1\\%[0.3em]
\vdots\ \\%[0.3em]
\varphi^n
\end{pmatrix}.
$$
The semantics of the vectorial formula $\vec\varphi_A$ for a Markov chain $M$ 
and a valuation $\rho$ is the semantics of its first component over the initial 
location $s^{in}$ of $M$ and it is equivalent to the value of the configuration 
$\left(s^{in}, t\left(\varphi^{in}\right)\right)$ in the semantics game 
$G_{M,\vec\varphi_A}$.
That is, $\llbracket \vec\varphi_A\downarrow_1\rrbracket^\rho_M(s^{in})\ =\ 
val_{G_{M,\vec\varphi_A}}\left(s^{in},t\left(\varphi^{in}\right)\right)$.

It is worth noticing that only a maximum of $n$ out of 
$m\times(n+1)$ variables are bound within the formula $\vec\varphi_A$.
Therefore, $\vec\varphi_A$ can be seen as a system of $n+1$ equations in $n$ 
variables that can be reduced by substitution and Gauss elimination techniques 
to a single scalar $\mu^p$-calculus sentence (see \cite{Sch04}).
In particular, it is sufficient to derive a solution, or expression, for each 
of the $n$ variables and by syntactical substitution embed such expressions in 
$t(\varphi^{in})$.
However, we are interested in giving a characterization in terms of semantics 
game $G_{M,\vec\varphi_A}$, in which the effect of the syntactical substitution 
of variables is handled by the edges connecting configurations $(s, X^k)$ to 
$(s,\varphi^k)$.
%
\begin{theorem}\label{the:PautToMup}
	Let $A$ be a p-automaton over the set $AP$ of atomic propositions and 
	$\vec\varphi_A$ the vectorial $\mu^p$-calculus formula resulting from its 
	conversion.
	Then, $A$ and $\vec\varphi_A$ are equivalent: the set of Markov chains that 
	constitute the language $\mathcal{L}(A)$ recognised by the p-automaton $A$ 
	coincides with the set of Markov chains that satisfy the vectorial formula 
	$\vec\varphi_A$.
	That is, $M\!\in\!\mathcal{L}(A)$ \emph{iff} $M\!\models\!\vec\varphi_A$.
\end{theorem}
Similarly to the case of the inverse translation, the proof %in Appendix 
%\ref{app:second}
shows that the semantics game $G_{M,\vec\varphi_A}$ for the 
vectorial formula $\vec\varphi_A$ simulates the acceptance game $G_{M,A}$ for 
the original p-automaton $A$.
As a result, the two games have the same value and, therefore, the Markov 
chains that satisfy the formula $\vec\varphi_A$ are exactly those that are 
accepted by $A$.

\section{Conclusion}
The aim of this paper was to investigate the connection between 
$\mu^p$-calculus and p-automata and to assess their equivalence in expressive 
power.
We presented the notion of well-formed formulas as a necessary preliminary step 
for their translation into p-automata.
We showed that for every well-formed $\mu^p$-calculus sentence there exists an 
equivalent p-automaton that recognises exactly all the Markov chains that model 
the formula.
Conversely, we proved that for every p-automaton there is an equivalent 
$\mu^p$-formula, in vectorial syntax, that is satisfied by the same Markov 
chains 
that form the language of the p-automaton.

Throughout this work, obligation games have played a key linking role in 
defining the semantics of the structures resulting from the conversions and, 
therefore, proving the correctness of our claims.


\bibliography{myBibliography}
\end{document}